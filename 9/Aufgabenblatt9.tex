\documentclass{article}

\usepackage[ngerman]{babel}

\usepackage{amsmath}
\usepackage{amssymb}

\begin{document}

\section*{Aufgabe 9.4.1}
\[
\Delta_{N_{9.4a}} =
\begin{bmatrix}
1   &	-1  &	0   &	0   &	0   &	0 \\
0   &	0   &	1   &	-1  &	0   &	0 \\
0   &	0   &	-1  &	1   &	0   &	0 \\
0   &	0   &	0   &	0   &	-1  &	1 \\
0   &	1   &	0   &	0   &	-1  &	0
\end{bmatrix}
\]

\section*{Aufgabe 9.4.2}

\section*{Aufgabe 9.4.3}
Da es zwei Transitionen gibt, die Marken in das System einspeisen, aber dabei
keine Entfernen ist die Invariante, die man durch das l"osen des
Gleichungssystems findet immer 0. Daher ist das Netz strukturell nicht
beschr"ankt.

\section*{Aufgabe 9.4.4}
\[
\Delta_{N_{9.4b}} =
\begin{bmatrix}
1   &	-1  &	0   &	0   &	0   &	0 \\
0   &	0   &	1   &	-1  &	0   &	0 \\
0   &	0   &	-1  &	1   &	0   &	0 \\
0   &	0   &	0   &	0   &	-1  &	1 \\
0   &	1   &	0   &	0   &	-1  &	0 \\
-1  &	0   &	0   &	0   &	0   &	0 \\
0   &	-1  &	0   &	0   &	1   &   0 \\
0   &	0   &	0   &	0   &	1   &   -1\\
\end{bmatrix}
\]

\[
\begin{bmatrix}
1   &	-1  &	0   &	0   &	0   &	0 \\
0   &	0   &	1   &	-1  &	0   &	0 \\
0   &	0   &	-1  &	1   &	0   &	0 \\
0   &	0   &	0   &	0   &	-1  &	1 \\
0   &	1   &	0   &	0   &	-1  &	0 \\
-1  &	0   &	0   &	0   &	0   &	0 \\
0   &	-1  &	0   &	0   &	1   &   0 \\
0   &	0   &	0   &	0   &	1   &   -1\\
\end{bmatrix}
\cdot
\begin{bmatrix}
i_0  \\
i_1  \\
i_2  \\
i_3  \\
i_4  \\
i_5  \\
i_6  \\
i_7  \\
\end{bmatrix}
= 0
\]

Ja, dieses Netz erf"ullt nach dem Satz im Skript die Bedingungen f"ur
Strukturelle Beschr"anktheit.

\section*{Aufgabe 9.4.5}
Der Informatiker wollte beim urspr"unglichen Netz darstellen, dass beliebig
viele Kunden kommen k"onnten und beliebig viele Lieferungen ankommen k"onnten. 
\end{document}
