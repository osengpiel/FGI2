\documentclass{article}

\usepackage[utf8]{inputenc}
\usepackage[ngerman]{babel}

\usepackage{amssymb}
\usepackage{amsmath}
\usepackage{stmaryrd} % Für \leftrightarroweq

\usepackage{latexsym}
\usepackage{graphicx}

\usepackage{tikz}
\usetikzlibrary{arrows,automata}

\title{FGI2 Übungen Blatt 3}

\author{Oliver Sengpiel, 6322763 \\ Daniel Speck, 6321317 \\ Daniel
Krempels, 6424833}

\begin{document}

\maketitle

\section*{3.3}
$L(A_1) = (a^* + (ba^*b)) + ((a^* + (ba^*b))^* ba^*)$ \\
$L(A_2) = (a^*ba^*(ba^*b)^*a^*)$ \\
$L^\omega(A_1) = (a + ba^*b)^*(ba^\omega) + (a + ba^*b)^\omega$ \\
$L^\omega(A_2) = a^*b(a^* + (ba^*b))^\omega$ \\

\section*{3.4}
\underline{Beweis: $TS_s \leftrightarroweq TS_r \Rightarrow TS_r
\leftrightarroweq TS_s$}\\
Gegeben sei eine Bisimulationsrelation $\mathcal{B}_s$, so dass $TS_s
\leftrightarroweq TS_r$ gilt.\\
Dann gilt:
\begin{align*}
&\forall s_0 \in S_s^0 : \exists r_0 \in S_r^0 : (s_0,r_0) \in \mathcal{B}_s \\
\wedge & \forall r_0 \in S_r^0 : \exists s_0 \in S_s^0 : (s_0,r_0) \in \mathcal{B}_s
\end{align*}
Es gibt nun eine Bijektion $\mu : S_s \times S_r \to S_r \times S_{s}$ mit
$\mu((s,r)) = (r,s)$.
Es sei nun $\mathcal{B}_r = \mu(\mathcal{B}_s)$ dann gilt auch:
\begin{align*}
&\forall r_0 \in S_r^0 : \exists s_0 \in S_s^0 : (r_0,s_0) \in \mathcal{B}_r \\
\wedge & \forall s_0 \in S_s^0 : \exists r_0 \in S_r^0 : (r_0,s_0) \in
\mathcal{B}_r
\end{align*}
Analog kann man zeigen, dass auch die beiden weiteren Bedingungen für
Bisimilarität (siehe Definition 2.4 im Skript) durch die Bijektion $\mu$ auch
für $\mathcal{B}_r$ gelten.

\underline{$TS_1 \leftrightarroweq TS_2$:}\\
Es existiert die Bisimulationsrelation $\mathcal{B}$ mit
\[
    \mathcal{B} = \{ (P_0,Q_0), (P_1,Q_1), (P_2,Q_1), (P_3,Q_2), (P_0,Q_3)\}
\]
Aus dem Beweis oben gilt $TS_1 \leftrightarroweq TS_2 \Rightarrow TS_2
\leftrightarroweq TS_1$.

\underline{$TS_1 \leftrightarroweq TS_3$:}\\
Die Transitionssysteme sind nicht bisimilar, da Bedingung b) bei dem Paar
$(P_1,R_1)$ verletzt ist.
Aus der Transitivität der Bisimilarität (zu $(s,r) \in \mathcal{B}_1$ und $(s,t)
\in \mathcal{B}_2$ gibt es ein $(r,t) \in \mathcal{B}_3$) folgt, dass $TS_3$
auch nicht bisimilar zu $TS_2$ ist.

\underline{$TS_1 \leftrightarroweq TS_4$:}\\



\end{document}
