\documentclass{article}

\usepackage[utf8]{inputenc}
\usepackage[ngerman]{babel}

\usepackage{amssymb}
\usepackage{amsmath}
\usepackage{stmaryrd} % Für \leftrightarroweq

\usepackage{latexsym}
\usepackage{graphicx}

\usepackage{tikz}
\usetikzlibrary{arrows,automata}

\title{FGI2 Übungen Blatt 3}

\author{Oliver Sengpiel, 6322763 \\ Daniel Speck, 6321317 \\ Daniel
Krempels, 6424833}

\begin{document}

\maketitle

\section*{3.3}

	\subsection*{3.3.1}
		$L(A_1) = (a^* + (ba^*b)) + ((a^* + (ba^*b))^* ba^*)$ \\
		$L(A_2) = (a^*ba^*(ba^*b)^*a^*)$ \\
		$L^\omega(A_1) = (a + ba^*b)^*(ba^\omega) + (a + ba^*b)^\omega$ \\
		$L^\omega(A_2) = a^*b(a^* + (ba^*b))^\omega$ \\

	\subsection*{3.3.2}
		\begin{center}
			\begin{tikzpicture}[auto, node distance=2cm,initial text={}]
			  \node[initial, state]		(qs)				{$qs$};
			  \node[state]				(rt) [right of=qs]	{$rt$};
			  \node[state, accepting]	(pt) [above of=rt]	{$pt$};

			 
			  \draw          (qs)  edge [->, loop above]              node {a} (qs);
			  \draw          (qs)  edge [->, bend left]             node {b}    (rt);
			  \draw			(rt)	edge [->, loop above]			node{a} 	(rt);
			  \draw 		(rt)	edge [->, bend left]			node{b}		(qs);
			  \draw			(qs)	edge [->]			node{b} 	(pt);
			  \draw			(pt)	edge [->, loop above]			node{a} 	(pt);
			\end{tikzpicture}
		\end{center}

\section*{3.4}
\underline{Beweis: $TS_s \leftrightarroweq TS_r \Rightarrow TS_r
\leftrightarroweq TS_s$}\\
Gegeben sei eine Bisimulationsrelation $\mathcal{B}_s$, so dass $TS_s
\leftrightarroweq TS_r$ gilt.
\end{document}