\documentclass{article}

\usepackage[utf8]{inputenc}
\usepackage[ngerman]{babel}

\usepackage{amssymb}
\usepackage{amsmath}
\usepackage{stmaryrd} % Für \leftrightarroweq

\usepackage{latexsym}
\usepackage{graphicx}

\usepackage{tikz}
\usetikzlibrary{arrows,automata}

\title{FGI2 Übungen Blatt 3}

\author{Oliver Sengpiel, 6322763 \\ Daniel Speck, 6321317 \\ Daniel
Krempels, 6424833}

\begin{document}

\maketitle

\section*{3.3}

	\subsection*{3.3.1}

		$L(A_1) = (a^* + (ba^*b)) + ((a^* + (ba^*b))^* ba^*)$ \\
		$L(A_2) = (a^*ba^*(ba^*b)^*a^*)$ \\
		$L^\omega(A_1) = (a + ba^*b)^*(ba^\omega) + (a + ba^*b)^\omega$ \\
		$L^\omega(A_2) = a^*b(a^* + (ba^*b))^\omega$ \\


	\subsection*{3.3.2}

		\begin{center}
			\begin{tikzpicture}[auto, node distance=2cm,initial text={}]
			  \node[initial, state]		(qs)				{$qs$};
			  \node[state]				(rt) [right of=qs]	{$rt$};
			  \node[state, accepting]	(pt) [above of=rt]	{$pt$};

			  \draw         (qs)    edge [->, loop above]           node{a} 	(qs);
			  \draw         (qs)    edge [->, bend left]            node{b}     (rt);
			  \draw			(rt)	edge [->, loop above]			node{a} 	(rt);
			  \draw 		(rt)	edge [->, bend left]			node{b}		(qs);
			  \draw			(qs)	edge [->]						node{b} 	(pt);
			  \draw			(pt)	edge [->, loop above]			node{a} 	(pt);
			\end{tikzpicture}
		\end{center}


	\subsection*{3.3.3}

		$L(A_3) = (a^* + ba^*b)^* ba^* = L(A_1) \cap L(A_2)$
		\\
		\\
		$L^{\omega}(A_3) = (a^* + ba^*b)^* ba^{\omega} \neq (a^* b a^* b)^{\omega} + (a^* + ba^*b)^{*} ba^{\omega}$  
		$= L^{\omega}(A_1) \cap L^{\omega}(A_2)$


	\subsection*{3.3.4}
		\begin{center}
			\begin{tikzpicture}[auto, node distance=3cm,initial text={}]
			  \node[initial, state, accepting]		(qs1)						{$qs1$};
			  \node[state]							(qs2) [right of=qs1]		{$qs2$};
			  \node[state]							(rt1) [below of=qs1]		{$rt1$};
			  \node[state]							(rt2) [right of=rt1]		{$rt2$};
			  \node[state]							(pt2) [above of=qs1]		{$pt2$};
			  \node[state, accepting]				(pt1) [right of=pt2]		{$pt1$};

			  \draw         (qs1)    edge [->]           			node{b} 	(pt2);
			  \draw         (pt2)    edge [->, bend left]           node{a} 	(pt1);
			  \draw         (pt1)    edge [->, bend left]           node{a} 	(pt2);
			  \draw         (qs1)    edge [->]           			node{a} 	(qs2);
			  \draw         (qs2)    edge [->, loop above]          node{a} 	(qs2);
			  \draw         (qs2)    edge [->]           			node{b} 	(rt2);
			  \draw         (rt2)    edge [->, bend left]           node{b} 	(qs1);
			  \draw         (qs1)    edge [->, bend left]           node{b} 	(rt2);
			  \draw         (rt2)    edge [->]           			node{a} 	(rt1);
			  \draw         (rt1)    edge [->, loop below]          node{a} 	(rt1);
			  \draw         (rt1)    edge [->]           			node{b} 	(qs1);
			  \draw         (qs2)    edge [->]           			node{b} 	(pt2);
			\end{tikzpicture}
		\end{center}

	\subsection*{3.3.5}

		$L(A_4) = (a^*ba^*b) + (a^* + ba^*b)^*ba + \epsilon \neq (a^* + ba^*b)^* ba^* = L(A_1) \cap L(A_2)$ 
		\\
		\\
		$L^{\omega}(A_4) = (a^* b a^* b)^{\omega} + (a^* + ba^*b)^{*} ba^{\omega}$  
		$= L^{\omega}(A_1) \cap L^{\omega}(A_2)$

\newpage

\section*{3.4}
\underline{Beweis: $TS_s \leftrightarroweq TS_r \Rightarrow TS_r
\leftrightarroweq TS_s$}\\
Gegeben sei eine Bisimulationsrelation $\mathcal{B}_s$, so dass $TS_s
\leftrightarroweq TS_r$ gilt.\\
Dann gilt:
\begin{align*}
&\forall s_0 \in S_s^0 : \exists r_0 \in S_r^0 : (s_0,r_0) \in \mathcal{B}_s \\
\wedge & \forall r_0 \in S_r^0 : \exists s_0 \in S_s^0 : (s_0,r_0) \in \mathcal{B}_s
\end{align*}
Es gibt nun eine Bijektion $\mu : S_s \times S_r \to S_r \times S_{s}$ mit
$\mu((s,r)) = (r,s)$.
Es sei nun $\mathcal{B}_r = \mu(\mathcal{B}_s)$ dann gilt auch:
\begin{align*}
&\forall r_0 \in S_r^0 : \exists s_0 \in S_s^0 : (r_0,s_0) \in \mathcal{B}_r \\
\wedge & \forall s_0 \in S_s^0 : \exists r_0 \in S_r^0 : (r_0,s_0) \in
\mathcal{B}_r
\end{align*}
Analog kann man zeigen, dass auch die beiden weiteren Bedingungen für
Bisimilarität (siehe Definition 2.4 im Skript) durch die Bijektion $\mu$ auch
für $\mathcal{B}_r$ gelten.

\underline{$TS_1 \leftrightarroweq TS_2$:}\\
Es existiert die Bisimulationsrelation $\mathcal{B}$ mit
\[
    \mathcal{B} = \{ (P_0,Q_0), (P_1,Q_1), (P_2,Q_1), (P_3,Q_2), (P_0,Q_3)\}
\]
Aus dem Beweis oben gilt $TS_1 \leftrightarroweq TS_2 \Rightarrow TS_2
\leftrightarroweq TS_1$.

\underline{$TS_1 \leftrightarroweq TS_3$:}\\
Die Transitionssysteme sind nicht bisimilar, da Bedingung b) bei dem Paar
$(P_1,R_1)$ verletzt ist.
Aus der Transitivität der Bisimilarität (zu $(s,r) \in \mathcal{B}_1$ und $(s,t)
\in \mathcal{B}_2$ gibt es ein $(r,t) \in \mathcal{B}_3$) folgt, dass $TS_3$
auch nicht bisimilar zu $TS_2$ ist.

\underline{$TS_1 \leftrightarroweq TS_4$:}\\



\end{document}
