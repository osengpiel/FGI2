\documentclass{article}

\usepackage[utf8]{inputenc}
\usepackage[ngerman]{babel}

\usepackage{amssymb}
\usepackage{amsmath}

\usepackage{latexsym}
\usepackage{graphicx}

\title{FGI2 Übungen Blatt 1}

\author{Oliver Sengpiel, 6322763 \\ Daniel Speck, 6321317 \\ Daniel
Krempels, 6YYYYYY}

\begin{document}

\maketitle
\section*{1.3}
\subsection*{1.3.1} 
$L(A_n)$ als regulärer Ausdruck:
\\
\\
$L(A_n) = (a^{2i}\cdot d\cdot b^{2i}) + (a^{2i-1}\cdot c\cdot
b^{2i-1}) + (a^{n}\cdot d) + (d) $ mit $i \in \{1, \dots, \frac{n}{2}
\}$

\subsection*{1.3.2}
$L(A_n)$ als Menge: \hfill $i \in \mathbb{N}$
\\
\\
$L(A_n) = \{ a^{2i} \cdot d \cdot b^{2i} \;|\; 0 \leq i \leq \frac{n}{2} \}
\cup
\{ a^{2i-1} \cdot c \cdot b^{2i-1} \;|\; 1 \leq i \leq \frac{n}{2} \}
\cup
\{ a^n \cdot d \}$

\subsection*{1.3.3}
%dachte ich fang mal an mit den nervigen plain-text sachen
Sei $M(A_n)$ genau die vom Automaten akzeptierte Sprache. \\ $L(A_n)
\subseteq M(A_n)$: 

$L(A_n)$ wird vom Automaten akzeptiert. Wird vom Startzustand aus eine gerade
Anzahl an 'a's gelesen, so erreicht man einen Zustand $p_i$ mit i mod 4 = 0.
Von hier aus kann das folgende 'd' gelesen werden, sowie dieselbe Anzahl an 'b's
wie 'a's, hiermit wird auch der Endzustand $p_1$ erreicht. Dies gilt auch für
$i=0$, hier wird ein einziges 'd' gelesen und man erreicht den Zustand $p_1$.
Wird eine ungerade Anzahl an 'a's gelesen, so wird ein Zustand $p_i$ mit i mod 4
= 2 erreicht. Von diesen aus kann das darauf folgende 'c' gelesen werden und
wiederum die selbe Anzahl an 'b's wie 'a's, und derselbe Endzustand $p_1$ wird
erreicht. Somit werden alle Eingaben von $L(A_n)$ akzeptiert. \\ $M(A_n)
\subseteq L(A_n)$: 

Alle vom Automaten akzeptierten Wörter sind in $L(A_n)$ enthalten.  Sei $w \in
M(A_n)$. $w$ kann drei verschiedene Formen haben: 1. $w$ kann aus einer
beliebigen, geraden Anzahl an 'a's, darauf folgend ein 'd' und darauf folgend
genau so viele 'b's wie 'a's bestehen. In diesem Fall gilt $w \in L(A_n)$. 2.
$w$ kann auch aus n 'a', gefolgt von einem 'd' bestehen. In diesem Fall gilt
ebenfalls $w \in L(A_n)$. 3. Oder $w$ ist aus einer ungeraden Anzahl an 'a's und
der gleichen Anzahl an 'b's aufgebaut, genau zwischen 'a's und b's ein 'c'. Auch
dieses $w$ ist in $L(A_n)$ enthalten. Somit sind alle Wörter, die vom Automaten
gelesen werden können, auch in $L(A_n)$, also gilt: $M(A_n) \subset L(A_n)$. 

\subsection*{1.3.4}

$L(A_n)$ ist regulär, da diese Sprache von einem endlichen Automaten akzeptiert
wird und (wie in Teilaufgabe 1.3.1 geschehen) durch einen regulären Ausdruck
beschrieben werden kann.

\subsection*{1.3.5}
 
$A = \bigcup_{n \geq 0} L(A_n)$ . Nun ist gegeben, dass n beliebig ist. Damit A
regulär ist, und also von einem endlichen Automaten akzeptiert wird, muss es
eine Schleife in einem Wort $w \in (a^{2i}\cdot c\cdot b^{2i})$ geben, bei dem
$i > n, n$ Anzahl der Zustände des Automaten. Doch sobald eine Schleife in dem
Wort auftritt, wird das Wort nicht mehr akzeptiert, da zum Beispiel nicht mehr
garantiert werden kann, dass genau so viele 'a's wie 'b's gelesen werden , oder
dass nur ein 'c' oder ein 'd' gelesen wird. Damit ist die Vereinigung aller
$L(A_n)$ für $n \geq 0$ nicht regulär. 

$A = \bigcup_{n \geq 0} L(A_n)$ ist aber kontextfrei, denn es lässt sich eine
Grammatik dazu aufstellen: \begin{align*} S &\rightarrow A | B \\ A &\rightarrow
aaAbb | acb | d \\ B &\rightarrow aaB | d \end{align*}
 
\section*{1.4}
\subsection*{1.4.1}

Sei $A := (Q, \Sigma, \delta, \{q^0\}, F)$ ein beliebiger endlicher Automat und
$A' := (Q', \Sigma', \delta', \{{q'}^0\}, F')$ ein Automat mit $L(A') = \{
    w^{rev} | w \in L(A)\}$, dann erhalten wir A' durch folgendes Verfahren: \\
\begin{enumerate}
\item \underline{Vertauschen des Anfangszustandes mit den Endzuständen:} \\
    Wir konstruieren zunächst einen NFA, der den Anfangszustand von A als
    Endzustand hat, und bei dem die Endzustände des Automaten A die
    Anfangszustände sind.
\item \underline{Umkehren der Übergangsrelationen:} \\
    Nun konstruieren wir die Übergangsrelation $\delta'$ mit $\delta' = \{(q,a,p)
    | (p,a,q) \in \delta \}$. Wir haben die Kanten unseres Automaten also alle
    umgedreht.
\item \underline{Erstellen des Potenzautomaten:} \\
    Um vom NFA wieder auf einen DFA zu kommen konstruieren wir nun den
    Potenzautomaten. Der Potenzautomat hat nun den alten Anfangszustand von A
    als Endzustand und einen Anfangszustand $q'$, der für alle Endzustände von A
    steht.
\end{enumerate}

Der Automat, den wir erhalten haben ist $A'$.

\subsection*{1.4.2}
$L(A) =\{e,f\}^*\{eef\}\{e,f\}^*$. Sei M(A) genau die Menge der vom Automaten
akzeptierten Wörter. 

$L(A) \subseteq M(A)$: L(A) wird von A akzeptiert. Von $ p_0$ aus können sowohl
beliebig viele 'f's als auch beliebig viele 'e's akzeptiert werden.  Werden nur
'f's gelesen, bleibt der Automat in $p_0$, werden nur 'e's gelesen, geht der
Automat erst in den Zustand $p_1$ und dann in den Zustand $p_2$ und liest dort
alle weiteren 'e's. Da der Automat vollständig ist, können von jedem Zustand aus
'e's und 'f's gelesen werden. Laut L(A) muss danach die Kombination {$eef$}
gelesen werden können. Unabhängig vom Zustand des Automaten vorher geht der
Automat nun nach $p_3$, den Endzustand. Hier können nun wieder beliebig viele
'e's und f's geesen werden, also wird L(A) von A azeptiert, und $L(A) \subseteq
M(A)$.   

$M(A) \subseteq L(A)$: Alle vom Automaten akzeptierten Wörter liegen in L(A).
Sei $w \ in M(A)$. $w$ kann aus beliebig vielen 'e's oder 'f's bestehen, mit der
Bedinung, dass irgendwo in dem Wort die Abfolge 'eef' vorkommt. Dies ist in L(A)
gegeben, damit gilt L(A) = M(A) und L(A) beschreibt genau die Menge der
akzptierten Wörter von A. 

\subsection*{1.4.3}
$L(A) = (e + f)* + (eef) + (e + f)*$

\subsection*{1.4.4}
Zunächst nehmen wir uns den Ausgangsautomaten A: \\
\begin{center}
\includegraphics[width=0.4\textwidth]{drawing.png}
\end{center}

Dann vertauschen wir Anfangs- und Endzustände des Automaten und erhalten
folgenden Automaten: \\
\begin{center}
\includegraphics[width=0.4\textwidth]{drawing22.png}
\end{center}

Im Schritt danach drehen wir die Übergangsrelation um:\\
\begin{center}
\includegraphics[width=0.4\textwidth]{drawing3.png}
\end{center}
\dots und sind fertig, da der Potenzautomat dieses Automaten wieder der Automat
selbst ist.


\subsection*{1.4.5}
Sei M(B) genau die Menge der vom Automaten B akzeptierten Wörter. Annahme: $L(B)
:= \{e,f\}^*\{fee\}\{e,f\}^* = M(B)$.

$L(B) \subseteq M(B)$: $L(B)$ wird von B akzeptiert. Von $p_3$ aus können sowohl
'e's als auch 'f's gelesen werden.  Generell kann von jedem Zustand aus ein 'e'
oder ein 'f' gelesen werden, da der Automat vollständig ist.  Laut $L(B)$ muss B
die Kombination 'fee' an einer Stelle des Worts lesen können. Unabhängig vom
Zustand des Automaten wechselt dieser nach Lesen von 'fee' in den Endzustand
$\{p_0, p_1, p_2, p_3\}$. Hier können beliebig viele weitere 'e's und 'f's von
$L(B)$ gelesen werden, damit akzeptiert der Automat $L(B)$.
 

$M(B) \subseteq L(B)$: Jedes Wort $w \in M(B)$ ist auch in $L(B)$. $w$ kann in
$p_3$ sowohl mit einem 'e' als auch mit einem 'f' beginnen. Bei fortlaufenden
'e's bleibt der Automat im Startzustand $p_3$, bei fortlaufenden 'f's wechselt
DFA B nach $\{p_2, p_3\}$ und verbleibt dort. Wird von dort aus ein 'e' gelesen,
wechselt der Automat zu $\{p_1, p_2, p_3\}$. Folgt in $w$ nun ein 'f', springt
der Automat zurück nach $\{p_2, p_3\}$. Um zu einem Endzustand zu gelangen, muss
$w$ irgendwo im Wort die Folge 'fee' beinhalten, damit der Automat in einen
Endzustand wechselt., Da der Automat nur in den Zustanden 'voranschreitet', wenn
ein solches Teilwort gelesen wird. $w$ kann also mit beliebig vielen 'e' und
'f's beginnen, muss die Abfolge {fee} beinhalten und kann mit beliebig
kombinierten 'e's und 'f's enden. Dies etspricht auch L(B), somit wird $w$ von
L(B) akzeptiert.


 
\end{document}

% d acb aadbb aaacbbb ... aaaaaaaaaa//aaaaad a^2i d b^2i + a^2i+1 c
% b^2i+1 + a^n d
