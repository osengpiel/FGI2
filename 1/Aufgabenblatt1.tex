\documentclass{article}

\usepackage[utf8]{inputenc} \usepackage[ngerman]{babel}

\usepackage{amssymb} \usepackage{amsmath}

\usepackage{latexsym}

\title{FGI2 Übungen Blatt 1}

\author{Oliver Sengpiel, 6322763 \\ Daniel Speck, 6321317 \\ Daniel
Krempels, 6YYYYYY}

\begin{document}

\maketitle
\setcounter{section}{2}
\section{}
\subsection{} 
$L(A_n)$ als regulärer Ausdruck: \\
$L(A_n) = (a^{2i}\cdot c\cdot b^{2i}) + (a^{2i-1}\cdot d\cdot
b^{2i-1}) + (a^{n}\cdot d) + (d)$ mit $i \in \{1, \dots, \frac{n}{2}
\}$

\subsection{}
%dachte ich fang mal an mit den nervigen plain-text sachen
Sei $M(A_n)$ genau die vom Automaten akzeptierte Sprache. \\ $L(A_n)
\subseteq M(A_n)$: 

$L(A_n)$ wird vom Automaten akzeptiert. Wird vom Startzustand aus
ein einziges 'd' gelesen, so geht der Automat direkt in den
Endzustand $p_1$ über und akzeptiert. Wird eine gerade Anzahl an
'a's gelesen, so erreicht man einen Zustand $p_i$ mit i mod 4 = 0.
Von hier aus kann das folgende 'd' gelesen werden, sowie dieselbe
Anzahl an 'b's wie 'a's, hiermit wird auch der Endzustand $p_1$
erreicht. Wird eine ungerade Anzahl an 'a's gelesen, so wird ein
Zustand $p_i$ mit i mod 4 = 2 erreicht. Von diesen aus kann das
darauf folgende 'c' gelesen werden und wiederum die selbe Anzahl an
'b's wie 'a's, und derselbe Endzustand $p_1$ wird erreicht. Somit
werden alle Eingaben von $L(A_n)$ akzeptiert. \\ $M(A_n) \subseteq
L(A_n)$: 

Alle vom Automaten akzeptierten Wörter sind in $L(A_n)$ enthalten.
Sei $w \in M(A_n)$. $w$ kann vier verschiedene Formen haben: 1. $w$
kann aus einem einzelnen 'd' bestehen, dies ist in $L(A_n)$
enthalten. 2. $w$ kann auch aus einer beliebigen, geraden Anzahl an
'a's, darauf folgend ein 'd' und darauf folgend genau so viele 'b's
wie 'a's bestehen. Auch in diesem Fall gilt $w \in L(A_n)$. 3. Oder
$w$ ist aus einer ungeraden Anzahl an 'a's und der gleichen Anzahl
an 'b's aufgebaut, genau zwischen 'a's und b's ein 'c'. Auch dieses
$w$ ist in $L(A_n)$ enthalten. Somit sind alle Wörter, die vom
Automaten gelesen werden können, auch in $L(A_n)$, also gilt:
$M(A_n) \subset L(A_n)$. 

\subsection{}

$L(A_n)$ ist regulär. Denn in der akzeptierten Sprache ist
festgelegt, dass kein Wort länger als $n + 1, n$ Anzahl der Zustände
sein kann. Damit kann es keine Schleifen in dem Wort geben und das
Pumping Lemma nicht widerlegt werden.

 \subsection{}
 
 $A = \bigcup_{n \geq 0} L(A_n)$ . Nun ist gegeben, dass n beliebig
ist. Damit A regulär ist, und also von einem endlichen Automaten
akzeptiert wird, muss es eine Schleife in einem Wort $w \in
(a^{2i}\cdot c\cdot b^{2i})$ geben, bei dem $i > n, n$ Anzahl der
Zustände des Automaten. Doch sobald eine Schleife in dem Wort auftritt, 
wird das Wort nicht mehr akzeptiert, da zum Beispiel nicht mehr garantiert werden kann, dass genau so viele 'a's wie 'b's gelesen werden , oder dass nur ein 'c' oder ein 'd' gelesen wird. Damit ist die Vereinigung aller $L(A_n)$ für $n \geq 0$ nicht regulär. 

 $A = \bigcup_{n \geq 0} L(A_n)$ ist aber kontextfrei, denn es lässt sich eine Grammatik dazu aufstellen:
\begin{align*}
 S &\rightarrow A | B \\
 A &\rightarrow aaAbb | acb | d \\
 B &\rightarrow aaB | d 
\end{align*}
 
\section{}
\subsection{}

Sei $A := (Q, \Sigma, \delta, \{q^0\}, F)$. 
Dann ist \[ A\textrm' = (Q\textrm', \Sigma\textrm', 
\delta\textrm', \{q^0\textrm'\},F\textrm') \]
mit \[ Q\textrm' = 2^Q,\]\[ \Sigma\textrm' = \Sigma,\] 
\[ q^0\textrm' = F,\] \[ F\textrm' = \{M \in 2^Q | M \cap q^0 \neq 0\} \] und wir bilden erst
\[K_\delta := \{ (p,x, \delta(p,x)~|~p \in Q, x \in \Sigma, \delta(p,x)~ definiert)\},\]
bilden \[K\textrm'= \{(p\textrm',x,p)~|~ (p,x,p\textrm') \in K_\delta\}\] und dann
\[\delta(M,x)= \cup_{z\in M} \{z\textrm' \in Q\textrm' | (z,x,z\textrm') \in K\textrm'\}\]
Alle Zustände, die vorher Endzustände waren, werden nun zu Startzuständen, alle vorherigen Startzustände werden
zu Endzuständen. Alle Kanten werden umgedreht und anschließend wird der Potenzautomat gebildet, da es bei der
Umkehr der Kanten dazu kommen kann, dass der Automat nicht-deterministisch ist. 
  

\subsection{}
$L(A) =\{e,f\}^*\{eef\}\{e,f\}^*$. Sei M(A) genau die Menge der vom Automaten akzeptierten Wörter. 

$L(A) \subseteq M(A)$: L(A) wird von A akzeptiert. Von $ p_0$ aus können sowohl beliebig viele 'f's als auch beliebig viele 'e's akzeptiert werden.  
Werden nur 'f's gelesen, bleibt der Automat in $p_0$, werden nur 'e's gelesen, geht der Automat erst in den Zustand $p_1$ und 
dann in den Zustand $p_2$ und liest dort alle weiteren 'e's. Da der Automat vollständig ist, können von jedem Zustand aus 
'e's und 'f's gelesen werden. Laut L(A) muss danach die Kombination {$eef$} gelesen werden können. Unabhängig vom Zustand
 des Automaten vorher geht der Automat nun nach $p_3$, den Endzustand. Hier können nun wieder beliebig viele 'e's und f's geesen werden, also 
wird L(A) von A azeptiert, und $L(A) \subseteq M(A)$.   

$M(A) \subseteq L(A)$: Alle vom Automaten akzeptierten Wörter liegen in L(A). Sei $w \ in M(A)$. $w$ kann aus beliebig vielen
'e's oder 'f's bestehen, mit der Bedinung, dass irgendwo in dem Wort die Abfolge 'eef' vorkommt. Dies ist in L(A) gegeben,
 damit gilt L(A) = M(A) und L(A) beschreibt genau die Menge der akzptierten Wörter von A. 

\subsection{}
Sei M(B) genau die Menge der vom Automaten B akzeptierten Wörter. Annahme: $L(B) := \{e,f\}^*\{fee\}\{e,f\}^* = M(B)$.


$L(B) \subseteq M(B)$: $L(B)$ wird von B akzeptiert. Von $p_3$ aus können sowohl 'e's als auch 'f's gelesen werden. 
Generell kann von jedem Zustand aus ein 'e' oder ein 'f' gelesen werden, da der Automat vollständig ist. 
Laut $L(B)$ muss B die Kombination 'fee' an einer Stelle des Worts lesen können. Unabhängig vom Zustand 
des Automaten wechselt dieser nach Lesen von 'fee' in den Endzustand $\{p_0, p_1, p_2, p_3\}$. Hier 
können beliebig viele weitere 'e's und 'f's von $L(B)$ gelesen werden, damit akzeptiert der Automat $L(B)$.
 

$M(B) \subseteq L(B)$: Jedes Wort $w \in M(B)$ ist auch in $L(B)$. $w$ kann in $p_3$ sowohl mit 
einem 'e' als auch mit einem 'f' beginnen. Bei fortlaufenden 'e's bleibt der Automat im Startzustand 
$p_3$, bei fortlaufenden 'f's wechselt DFA B nach $\{p_2, p_3\}$ und verbleibt dort. Wird von dort aus 
ein 'e' gelesen, wechselt der Automat zu $\{p_1, p_2, p_3\}$. Folgt in $w$ nun ein 'f', springt der 
Automat zurück nach $\{p_2, p_3\}$. Um zu einem Endzustand zu gelangen, muss $w$ irgendwo im Wort 
die Folge 'fee' beinhalten, damit der Automat in einen Endzustand wechselt., Da der Automat nur 
in den Zustanden 'voranschreitet', wenn ein solches Teilwort gelesen wird. $w$ kann also mit beliebig 
vielen 'e' und 'f's beginnen, muss die Abfolge {fee} beinhalten und kann mit beliebig kombinierten 
'e's und 'f's enden. Dies etspricht auch L(B), somit wird $w$ von L(B) akzeptiert.


 
\end{document}

% d acb aadbb aaacbbb ... aaaaaaaaaa//aaaaad a^2i d b^2i + a^2i+1 c
% b^2i+1 + a^n d

