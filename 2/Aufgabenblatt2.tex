\documentclass{article}

\usepackage[utf8]{inputenc}
\usepackage[ngerman]{babel}

\usepackage{amssymb}
\usepackage{amsmath}

\usepackage{latexsym}
\usepackage{graphicx}

\title{FGI2 Übungen Blatt 1}

\author{Oliver Sengpiel, 6322763 \\ Daniel Speck, 6321317 \\ Daniel
Krempels, 6YYYYYY}

\begin{document}

\maketitle

\section*{2.3}
\subsection*{2.3.1}
Die Sprache $L(A_{2.3})$ ist genau die Sprache $L = \{ a\} \{ba^*c\}^* \cup
\{ b\} \{cab\}^* \{c\} \{e , a\}$. \\
Die Sprache $L^\omega(A_{2.3})$ ist genau die Sprache $L^\omega = \{a \} 
\{ba^*c\}^\omega \cup \{ b \}\{cab\}^\omega $. \\
Die Sprache $(L(A_{2.3}))^\omega$ ist genau die Sprache, die Wörter der Form
$((a (ba^*c)^*) + (b (cab)^* c (e + a)))^\omega$ enthält.

\section*{2.4}
\subsection*{2.4.2}
\begin{enumeration}
\item Man konstruiere einen NFA, der die Menge der Wörter $W$ akzeptiert.
\item Nun konstruiere man einen Büchiautomaten, der $U$ akzeptiert.
\item Als nächstes erhalten alle Endzustände des NFA die
Übergangsrelationen der Startzustände des Büchiautomaten (alle ein- und
ausgehenden Kanten).
\item Der alte Startzustand des Büchiautomaten wird entfernt.

\subsection*{2.4.3}
Zunächst gilt es zu zeigen, dass die Wörter, die unser konstruierter
Automat (nennen wir ihn $A$) akzeptiert in $W\cdot U$ liegen: \\

\underline{$L(A) \subseteq W\cdot U$:}\\
Für unser Wort $w$ gibt es einen Pfad durch den Automaten, der von einem
Startzustand des ehemaligen NFAs zu einem ehemaligen Endzustand des
selbigen führt. Gleichzeitig gibt es für jedes Wort einen Pfad, der von einem
Nachfolgerzustand des ehemaligen alleinstehenden Büchiautomaten zu einer
Schleife durch einen Endzustand führt. Da wir nun Kanten von den
ehemaligen Endzuständen des NFA zu den Nachfolgern der Anfangszustände des
Büchiautomaten konstruiert haben, die das gleiche Eingabesymbol lesen, wie
die alten Kanten des Anfangszustandes, sind die beiden Pfade miteinander
verbunden und ergeben einzelne, lange Pfade, die unendlich oft durch
Endzustände führen.
\end{document}
