\documentclass{article}

\usepackage[utf8]{inputenc}
\usepackage[ngerman]{babel}

\usepackage{amssymb}
\usepackage{amsmath}

\usepackage{latexsym}
\usepackage{graphicx}

\title{FGI2 Übungen Blatt 1}

\author{Oliver Sengpiel, 6322763 \\ Daniel Speck, 6321317 \\ Daniel
Krempels, 6YYYYYY}

\begin{document}

\maketitle

\section*{2.3}
\subsection*{2.3.1}
Die Sprache $L(A_{2.3})$ ist genau die Sprache $L = \{ a\} \{ba^*c\}^* \cup
\{ b\} \{cab\}^* \{c\} \{e , a\}$. \\
Die Sprache $L^\omega(A_{2.3})$ ist genau die Sprache $L^\omega = \{a \} 
\{ba^*c\}^\omega \cup \{ b \}\{cab\}^\omega $. \\
Die Sprache $(L(A_{2.3}))^\omega$ ist genau die Sprache, die Wörter der Form
$((a (ba^*c)^*) + (b (cab)^* c (e + a)))^\omega$ enthält.


\end{document}
