\documentclass{scrartcl}

\usepackage{eurosym}
\usepackage[utf8]{inputenc}
\usepackage[T1]{fontenc}
\fontfamily{ugm}
\usepackage{lmodern}
\usepackage[german]{babel}
\usepackage[babel]{csquotes}
\usepackage[german]{selnolig}
\usepackage{amsmath} 
\usepackage{amssymb}
\usepackage{amstext}
\usepackage{amsfonts}
\usepackage{amsbsy}
\usepackage{amscd}
\usepackage{amsopn}
\usepackage{amsxtra}
\usepackage{mathrsfs}
\usepackage{booktabs}
\usepackage{graphicx} 
\usepackage{tikz} 
\usetikzlibrary{arrows,automata}

\usepackage{caption}
\usepackage{dcolumn}
\usepackage{ulem}
\usepackage{multirow} 
\usepackage{bigstrut}

\title{Beispiel-Automaten}

\author{Mau}

\begin{document}

\maketitle

\section{}
\begin{center}
\begin{tikzpicture}[auto, node distance=2cm,initial text={}]
  \node[initial, state]   (q0)                          {$q_0$};
  \node[state]           (q3) [right of=q0]				{$q_3$};
  \node[state]           (q4) [right of=q3]             {$q_4$};
  \node[state,accepting]           (q5) [below of=q3, right of=q3] {$q_5$};
%  \node[state,accepting] (D) [right of=q3, below of=q0] {`$D$`};

 
  \draw          (q0)  edge [->]              node {b} (q3);
  \draw          (q3)  edge [->]             node {c}    (q4);
  \draw          (q4)  edge [->]             node {a}    (q5);
%  \path[->]          (q4)  edge [loop left]  node {0,1}  (q4);
%  \path[->]          (q5)  edge              node {0}    (D);
  \draw          (q5)  edge [->]             node {b}    (q3);
%  \path[->]          (D)  edge              node {0,1}  (q0);
\end{tikzpicture}
\captionof{figure}{Beispiel-Automat}\label{1}
\end{center}
\section{}
\begin{center}
\begin{tikzpicture}[auto, node distance=2cm,initial text={}]
  \node[initial, state]   (p0)                          {$p_0$};
  \node[state]           (p1) [right of=p0]				{$p_1$};
  \node[state]           (p2) [right of=p1]             {$p_2$};
  \node[state]           (p3q0) [right of=p2] {$p_3q_0$};
%  \node[state]   (q0) [right of=p3]             {$q_0$};
  \node[state, accepting]   (q1) [below of=p3q0]             {$q_1$};
    \node[state]   (q2) [right of=q1]             {$q_2$};
  \node[state]           (q3) [right of=p3q0]				{$q_3$};
  \node[state]           (q4) [right of=q3]             {$q_4$};
  \node[state,accepting]           (q5) [below of=q3, right of=q3] {$q_5$};
    \node[state,accepting]           (q6) [ right of=q4] {$q_6$};
%  \node[state,accepting] (D) [right of=q3, below of=q0] {`$D$`};

 \draw          (p0)  edge [->, loop above]   node {f} (p0);
 \draw          (p0)  edge [->, bend left]              node {e} (p1);
 \draw          (p1)  edge [->, bend left]              node {f} (p0);
 \draw          (p1)  edge [->]              node {e} (p2);
 \draw          (p2)  edge [->, loop above]              node {e} (p2);
 \draw          (p2)  edge [->]              node {f} (p3q0);
 \draw          (p3q0)  edge [->, loop above]              node {e,f} (p3q0);
% \draw          (p3)  edge [->]              node {$\epsilon$} (q0); 
 \draw          (p3q0)  edge [->]              node {b} (q3);
 \draw          (p3q0)  edge [->]   node {a} (q1);
 \draw          (q1)  edge [->, bend left]   node {b} (q2);
 \draw          (q2)  edge [->, bend left]   node {c} (q1);
 \draw          (q2)  edge [->, loop below]   node {q} (q2);
  \draw          (q3)  edge [->]             node {c}    (q4);
  \draw          (q4)  edge [->]             node {a}    (q5);
%  \path[->]          (q4)  edge [loop left]  node {0,1}  (q4);
%  \path[->]          (q5)  edge              node {0}    (D);
  \draw          (q5)  edge [->]             node {b}    (q3);
%  \path[->]          (D)  edge              node {0,1}  (q0);
 \draw          (q4)  edge [->]   node {e} (q6);
\end{tikzpicture}
\captionof{figure}{Beispiel-Automat} \label{2}
\end{center}
\end{document}
